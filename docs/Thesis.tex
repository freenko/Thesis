\documentclass[11pt]{article}
\usepackage{graphicx}
\usepackage{microtype}
\usepackage[hidelinks]{hyperref}
\usepackage{xurl}
\sloppy

      \textwidth 15cm
      \textheight 22cm
      \parindent 10pt
      \oddsidemargin 0.85cm
      \evensidemargin 0.37cm


\begin{document}

\thispagestyle{empty}

\begin{center}

Vrije Universiteit Amsterdam

\vspace{1mm}

\includegraphics[height=28mm]{vu_symbol.png}

\vspace{1.5cm}

{\Large Bachelor Thesis}

\vspace*{1.5cm}

\rule{.9\linewidth}{.6pt}\\[0.4cm]
{\huge \bfseries Machine Learning-Driven Process Anomaly Detection 
in Pharmaceutical Batch Manufacturing\par}
\rule{.9\linewidth}{.6pt}\\[1.5cm]

\vspace*{2mm}

{\Large
\begin{tabular}{l}
{\bf Author:} ~~Franko Muhametaj ~~~~ (2777133)
\end{tabular}
}

\vspace*{1.5cm}

\begin{tabular}{ll}
{\it 1st supervisor:}   & ~~Prof. Dr. Wan Fokkink \\
{\it daily supervisor:} & ~~Gerard McNelis ~~~~ \\
{\it 2nd reader:}       & ~~Dr. Natalia Silvis-Cividjian
\end{tabular}

\vspace*{2cm}

\textit{A thesis submitted in fulfillment of the requirements 
for\\ the VU Bachelor of Science degree in Computer Science }

\vspace*{1cm}

\today\\[4cm] % Date

\end{center}

\newpage

\twocolumn[
\begin{center}
    \Large \textbf{Machine Learning-Driven Process Anomaly 
    Detection in Pharmaceutical Batch Manufacturing}
\end{center}
\vspace{1em} 
]

\section*{Abstract}
Pharmaceutical production plays a vital role in improving global 
health by delivering high-quality products reliably. Early fault detection in manufacturing 
is critical for product quality, waste reduction, and regulatory compliance.

This thesis investigates whether machine learning models are able to detect
and catch early faults during the run of a batch process using a smaller set of sensors.
The IndPenSim simulated dataset was used for this experiment, which contained data for a total of 100 batches. Two complementary machine learning methods were applied: 
Gradient Boosted Trees (GBT) were used to identify and rank the sensors that contributed 
the most to the fault of a batch, achieving 90.6\% accuracy. The top 5 sensors formed the 
basis for a reduced monitoring setup with the LSTM.

Long Short-Term Memory (LSTM) networks were trained on normal batches to learn typical 
process behavior and detect faulty runs. The model reached a 3.7\% false alarm rate and 
detected 43.1\% of post-fault sequences, providing an early warning of 3-4 hours before 
batch 
completion. While the results are moderate, the approach lowers monitoring complexity and 
still offers some valuable insights.

This research also discusses how to implement these models into an ISA-95 architecture from 
a theoretical standpoint, highlighting where the neural network and the batch-level 
classifier could operate in practice. This section outlines the need to move from a simulation to a regulated environment.

In conclusion, the findings suggest that a targeted, cost-effective strategy can deliver valuable insights to pharmaceutical environments. This work accentuates the trade-off
between performance and deployment by providing a foundation for future work on machine learning and the pharmaceutical world.

\section{Introduction\&Background}

\noindent Source code and notebooks are available at \url{https://github.com/freenko/Thesis}.\footnote{\url{https://github.com/freenko/Thesis}}

\subsection*{Context and motivation}

This thesis was motivated by the author's experience in a large, regulated 
pharmaceutical manufacturing environment where Manufacturing 
Execution Systems (MES) coordinate batch execution,
electronic batch records, and compliance-relevant workflows 
across the site.During the internship, the author contributed to MES-related activities such as supporting and executing
MES projects, creating controlled documentation, and preparing user-facing training materials for operators and engineers.
A practical observation from this setting is that 
industrial manufacturing systems are diverse from site to site: they often share similar 
objectives; however, they can differ in equipment, instrumentation, and 
data pipelines. As a result, it is very difficult to develop a solution that can be adapted
across multiple sites. 
This thesis explores a fault detection approach with a reduced set of sensors that keeps 
the deployment constraints in mind. 
The goal is to support the pharmaceutical industry, in particular
, shop-floor operators and engineers, by reducing monitoring complexity and providing 
a possible implementation of this framework into manufacturing system layers.

\subsection{Problem Statement}

Pharmaceutical manufacturing is held to incredibly high standards 
because the products directly affect patient health. Even small 
deviations in the production process can lead to failed batches, 
safety risks, and millions of euros in losses [9]. Batch 
processes, such as penicillin production, require monitoring dozens of variables, such as temperature, pH, dissolved 
oxygen, and flow rates, to ensure quality [1, 11].

The problem is that current approaches to catching faults have 
serious limitations. First, comprehensive monitoring is expensive. 
Industrial pharmaceutical sensors cost a lot to install and maintain, and every additional sensor adds complexity to the system [1, 7]. Second, traditional alarm systems create too 
many false alerts, causing operators to start ignoring 
warnings [2]. 
Third, by the time lab tests confirm that something went wrong, the 
batch is usually already complete and must be discarded.

This creates a challenging situation: pharmaceutical companies 
need better fault detection that works with fewer sensors, catches 
problems earlier in the process, and actually functions in practice. 
The challenge is especially difficult because in real production, faulty 
batches are rare compared to normal ones, making it hard for 
detection systems to learn what faults look like [4, 6].

\subsection{Machine Learning for Fault Detection}

Machine learning can offer a promising solution to these challenges. 
Machine learning algorithms can learn complex patterns from historical data and 
identify subtle warning signs that traditional systems miss [2, 
6]. 
Recent research shows that these approaches can maintain high 
detection accuracy while using far fewer sensors than conventional 
monitoring [5, 6]. Soft sensor techniques using 
machine learning can also estimate difficult-to-measure process 
variables in real-time, reducing the need for expensive physical 
sensors [12].

\indent
Batch processes that involve biological material 
evolve through phases (growth, feeding, stabilization), meaning that
a sensor measurement can have different values throughout a batch run. Many faults have
temporal signatures such as: gradual drifts (pH for example), delayed
responses, or phase shifts where the flow deviates from the expected course. Methods
that model temporal dependencies can provide earlier warnings signals instead of end-of-batch information.

Two types of algorithms are particularly useful for this problem. 
Gradient Boosted Trees are excellent at figuring out which sensors 
matter for detecting faults. They analyze 
historical data and rank each sensor based on how much it contributes to 
distinguishing normal batches from faulty ones [5]. This helps 
answer a critical question: which measurements are have the highst impact in a batch run?

Long Short-Term Memory networks are very effective for capturing how features develop over time. Unlike traditional 
models that treat each measurement independently, LSTMs understand 
that what happens at hour 5 of a batch can affect what happens at 
hour 15 for example[3, 4]. This temporal understanding is crucial for batch 
processes where conditions evolve throughout production.

However, applying these techniques to pharmaceutical manufacturing 
presents specific challenges. The biggest issue is class imbalance;
in a typical dataset, you might have 90 normal batches and only 10 
faulty ones like the one used in this experiment [13]. Without special handling, models trained on this 
data just learn to predict "normal" for everything. Various 
approaches address this challenge, including 
synthetic oversampling techniques, class weighting, 
and alternative problem formulations like anomaly detection where 
models learn only from normal data and flag deviations. 
Pharmaceutical regulations also require that any 
automated decision system be explainable and validated, which adds 
constraints that some machine learning approaches struggle to meet [1, 9].

\subsection{Industry 4.0 and Manufacturing Systems Integration}

Developing a good algorithm is only half the challenge. The other 
half is figuring out how it actually gets deployed in a real 
pharmaceutical facility. Modern manufacturing plants operate 
within a structured framework called ISA-95, which organizes different 
systems into levels [8, 10].

At the bottom level, you'll find sensors and equipment that interact directly 
with the process. Above that, SCADA and PLC systems handle real-time control like adjusting temperatures and flow rates. Then 
comes the Manufacturing Execution System (MES), which coordinates 
production schedules, tracks materials, and manages quality 
records [8, 10]. 
At the top, business systems handle activities like inventory and 
financial planning.

Machine learning fault detection models fit most 
naturally at the SCADA/PLC level where they can access real-time 
sensor data and make immediate predictions [8]. But for these 
predictions to be useful, they need to integrate properly with the 
MES that actually coordinates what happens in the plant [10]. 
Recent research on pharmaceutical MES systems shows that this 
integration is possible, but there's still a gap between what 
works in research papers and what works in validated manufacturing 
environments [8, 9].

An emerging approach to bridge this gap is with Machine 
Learning Operations (MLOps). MLOps practices address critical operational 
challenges like model versioning, performance monitoring, 
automated retraining, and continuous validation, all essential for 
pharmaceutical applications where regulatory compliance
and reliability are very important. This topic
is not fully covered by academic research, but is crucial for 
transitioning from experimental prototypes to production-ready systems.

\subsection{Research Objectives and Contributions}

This thesis aims to answer a specific question: \textit{How 
effectively can machine learning models detect and
explain process anomalies in pharmaceutical batch
manufacturing based on multivariate sensor data?}

\noindent In this context, effectiveness is measured by identifying the minimal sensor 
infrastructure that maintains acceptable early warning capability. This reflects a 
fundamental constraint: comprehensive monitoring is expensive, requiring investment in sensors, data infrastructure, system 
integration, and ongoing maintenance [1, 7]. The research question therefore addresses a 
deliberate trade-off between detection performance and monitoring complexity. Can a 
reduced sensor set provide useful early warnings with acceptable false alarm rates, even 
if detection rates are modest compared to using all available sensors? This optimization 
aligns with industrial reality, where deployment decisions must balance multiple factors including cost, 
regulatory validation, system reliability. The question 
addresses both technical capability (which sensors matter, when do faults occur, what 
detection rates are achievable) and operational feasibility (can this work in practice with 
reduced infrastructure while maintaining acceptable performance).

Using simulated data from 100 batches of penicillin production 
(the IndPenSim platform) [3], this research has three main goals:

\textbf{First}, compare two complementary machine learning approaches 
for fault detection: Gradient Boosted Trees (GBT) for batch-level fault 
classification and Long Short-Term Memory (LSTM) networks for 
real-time anomaly detection. The goal is to determine which approach 
performs best for different aspects of fault detection and whether 
they can be effectively combined.

\textbf{Second}, use the LSTM model to learn normal process behavior from fault-free batches and detects 
anomalies by identifying when sensor predictions deviate significantly 
from expected patterns. This approach provides temporal localization 
of faults, identifying when values deviates from acceptable ranges. The focus is on maintaining 
early warning capability with a reduced sensor set rather than maximizing detection rates 
with comprehensive instrumentation.

\textbf{Third}, analyze how these models would actually be 
deployed in a pharmaceutical plant. This means examining integration within 
the ISA-95 framework, understanding how fault detection at the 
control level interfaces with MES coordination, and identifying 
practical challenges for moving from research to validated 
production systems [12].

The research makes two key contributions. It first
demonstrates that just 5 selected sensors show some early warning capability 
3-4 hours before batch completion with 43.1\% sequence-level detection and 3.7\% false 
alarm rate. This represents a 74\% reduction in sensor infrastructure compared to using all 
19 online sensors. The trade-off addresses practical constraints of analysis with reduced sensor sets. 
The GBT approach achieves 90.6\% batch-level classification accuracy 
through cross-validated feature selection, while the LSTM provides 
real-time anomaly detection by learning normal process behavior from 
fault-free batches. The second contribution provides an analysis of 
deployment considerations critical for actual industrial implementation 
[8, 10, 11, 12].

\paragraph{Key findings:} 
In the experiments on simulated penicillin fermentation batches, Gradient Boosted Trees
provide a stable sensor ranking and strong batch-level discrimination under severe class 
imbalance, while the LSTM forecasting approach can flag deviations several hours before
batch completion when monitoring is restricted to the most informative sensors. The main
practical take away is that a reduced sensor set can still provide useful early warning signals.

\section{Related Work}
\label{sec:related_work}

Papers on machine learning for industrial fault detection focus mainly on two areas: methods
that classify batches as normal or faulty, and methods that try to learn how the process evolves through time.
Many reviews highlights the fact the real industrial settings have few faulty examples, measurements
usually have noise and that any monitoring solution must be interpretable and 
practical enough for operators to understand [2,6].

In pharmaceutical and bioprocess settings, these challenges are more apparent. The fermentation dataset used shows how complicated
batch processes can be, with many variables that change over time [13].
This has pushed researchers and experts into exploring how machine learning models can learn
how variables relate to each other during different process phases.
Other studies have applied predictive models to detect equipment failure, reporting strong accuracy but
under different assumptions compared to the one used in this project [1].

From a modeling perspective, tree-based methods showed strong performance for batch classification
with structured data. They handle well mixed features and they provide interpretable outputs such as feature importance rankings [5].
In parallel, deep learning models have been effective in capturing temporal dependencies in time-series data.
Neural networks are usually used for direct prediction or for forecasting future values with sensors, where 
deviations from expected patterns indicate anomalies [3,4].

Another important aspects looks at how these models can be implemented into modern manufacturing systems.
Industry 4.0 frameworks like ISA-95 highlights the push for digitalization, increased data integration, and real-time analytics
and constraints introduced by regulated environments [8,9,10].
Reviews of manufacturing execution systems describes how analytics can fit into existing workflows, without disrupting the process [11,12].
A more recent contribution frame this integration challenge through MLOps practices [12], which address operational challenges like model versioning, performance monitoring,
and continuous validation, all essential for the pharmaceutical context.
In this thesis, the ISA-95 hierarchy provides the architecture for situating the proposed machine learning models within a realistic manufacturing system [15].

This thesis contributes to the existing literature by combining these aspects: it applies both tree-based and deep learning methods to a realistic pharmaceutical batch dataset
and it explicitly examines the trade-off between detection performance and monitoring complexity.
The results add to the existing literature by showing how sensor reduction and early-warning can be brought
together in a single workflow, while also analyzing practical deployment considerations within the ISA-95 framework and 
the costraints of limited faulty data [11,13].

\section{Methodology}
\subsection{Dataset}
In this reasearch, the IndPenSim dataset is used [13]. It's a simulated dataset that mimics
the penicillin fermentation process, which is a good example of a common pharmaceutical process and it
replicates well the biological conditions found in real production.
The dataset contains 100 batches, of which 90 are noraml runs and 10 contain faults. Each batch consistso of
time-series data collected from 2,039 features at regular intervals of 12 minutes, covering the entire batch duration.

Although the number of run is limited, the IndPenSim dataset is designed and published as a benchmark for data-driven monitoring
and fault detection research in pharmaceutical batch processes [13], making it valuable for machine learning studies.
The V3 file used in this paper contains 113{,}913 time-stamped rows across all batches and 2,239 columns
Each row represents a single time point within a batch and it inlcudes a wide range of sensor measurements,
while each column represents a specific sensor, specifically a distinct feature.
Most features are Raman spectroscopy measurements, while others are process variables like temperature, pH, and concentrations of various substances.

This structure results in a rich multivariate dataset that supports both batch-level
classification and time-series forecasting approaches for fault detection.
It offers a realistic high-frequency, multivariate time-series represantion of bioprocess dynamics,
fault injection that enables supervised learning, and a reproducible benchmark for evaluating multiple algorithms.
This dataset is also suitable because it reflects practival challenges: faults are rare due to the high impact of failed batches in real production,
process varibales are strongly correlated over time, and sensors contains some level of noise which confirms
that this dataset represents a realistic environment. 

As mentioned earlier, the full dataset contains over 2,000 features per batch, 2,000+ Raman variables
and approximately 40 process variables. For this analysis, only 19 online sensors were
kept, as Raman spectroscopy and other measurements are not typically available in real-time industrial settings.

The strong class imbalance reflects real conditions, failures are rare but have high impact. 
As shown in Figure~\ref{fig:classImbalance}, the dataset has a 9:1 ratio between normal and faulty batches. 
This imbalance required dedicated handling strategies, discussed later in Section~\ref{sec:gbt} and in Section~\ref{sec:lstm}.

\begin{figure}[htbp]
\centering
\includegraphics[width=\columnwidth]{classImbalance.png}
\caption{Class distribution showing severe imbalance between normal (90 batches) and
faulty (10 batches) operations.}
\label{fig:classImbalance}
\end{figure}

The imbalance is especially important in fault‑detection settings because:

\begin{itemize}
\item \textbf{Missing a fault (False Negative)} can have serious consequences, including batch failure, contamination, equipment damage, and major financial losses.
\item \textbf{False alarms (False Positives)} are disruptive but less severe, as they trigger unnecessary interventions rather than catastrophic failures.
\end{itemize}

\subsection{Data preparation}
Raw time-series data from the IndPenSim V3 dataset underwent systematic preprocessing to 
ensure data quality and prepare it for model training. This preprocessing pipeline 
consisted of three key steps: data cleaning, normalization, and dataset splitting.

\subsubsection{Data Cleaning and Feature Selection}

The original dataset contained 2,039 features per batch, including 2,000+ Raman 
spectroscopy measurements and various process variables. However, Raman spectroscopy 
requires expensive equipment and provides measurements that are not available in real-time 
industrial settings. The analysis focused exclusively on the 19 online sensors 
that would be available during actual batch production.
Four offline measurements were excluded even though they appeared in the 
process variables section: penicillin concentration, biomass concentration, product 
concentration, and substrate concentration.

Missing values in the sensor data were rare and 
were handled using forward-fill and backward-fill imputation. This approach maintains 
temporal continuity without introducing artificial patterns that could mislead the models.


\subsubsection{Data Normalization}

Since different sensors measure different physical quantities with different units of measurement,
standardization was necessary to prevent scale-sensitive algorithms from being dominated by large-magnitude 
features.

Z-score normalization was applied to all sensor measurements:

\begin{equation}
   Z = \frac{x - \mu}{\sigma}
\end{equation}

\noindent where $\mu$ is the feature mean and $\sigma$ is the standard deviation calculated from 
the training set to prevent data leakage.

\subsubsection{Dataset Partitioning}

The partitioning strategy differed between the two modeling approaches:

\paragraph{Gradient Boosted Trees (Classification):}
The 100 batches were split using stratified random sampling to preserve the 10\% fault 
rate:
\begin{itemize}
    \item \textbf{Training set}: 70 batches (63 normal, 7 faulty)
    \item \textbf{Validation set}: 15 batches (14 normal, 1 faulty) 
    \item \textbf{Test set}: 15 batches (13 normal, 2 faulty) 
\end{itemize}

\paragraph{LSTM (Anomaly Detection):}
To learn normal process behavior, the 90 normal batches were partitioned as:
\begin{itemize}
    \item \textbf{Training set}: 63 normal batches (70\%)
    \item \textbf{Validation set}: 18 normal batches (20\%)
    \item \textbf{Test set}: 9 normal batches (10\%)
\end{itemize}

All 10 faulty batches were held out completely and used exclusively for anomaly detection 
evaluation for the LSTM experiment. By training the model only on normal batches, it learns the typical pattern
of normal operations. This establishes a baseline for detecting anomalies for when the model 
is tested with data that includes faults, if the model's predictions deviate significantly, it indicates a potential fault.

This strict separation prevents any leakage of fault information into the training process.
It also reflects real-world conditions where the model would monitor ongoing production without prior exposure to fault patterns.

\subsection{Feature Engineering}

\subsubsection{Gradient Boosting Trees}
\label{sec:gbt}

To establish a baseline for fault detection, Gradient Boosting Trees (GBT) were employed 
for their effectiveness in handling tabular data and identifying 
important features. Unlike traditional machine learning approaches that require manual 
feature extraction, GBT automatically learns which sensor measurements are most indicative 
of fault conditions.

The raw time-series data from each batch were transformed into statistical summaries, 
capturing the overall behavior of each sensor throughout the batch duration. For each 
sensor measurement, the following features were computed: mean, standard deviation, 
minimum, maximum, median, and interquartile range. This transformation converted variable-length time-series batches into fixed-length feature 
vectors suitable for tree-based classification.

The XGBoost implementation was selected for its computational efficiency and built-in 
regularization capabilities. The model was trained to distinguish between normal and 
faulty batches using binary classification, with the fault reference label as the target 
variable. 
Cross-validation was performed to assess model stability and prevent overfitting.

\begin{figure*}[htbp]
    \centering
    \includegraphics[width=0.9\textwidth]{GBT feature.png}
    \caption{Feature importance (a) Sensor rankings from single train-validation split. 
    (b) Mean importance across 5-fold cross-validation.}
    \label{fig:gbt_feature_stability}
\end{figure*}

\begin{figure*}[htbp]
    \centering
    \includegraphics[width=0.9\textwidth]{top sensors_across folds.png}
    \caption{Top 10 sensor importance rankings across 5-fold cross-validation. Individual 
    folds (left five panels) show minor variations in exact rankings, while the mean 
    importance (rightmost panel) identifies the most stable predictors. Eight sensors 
    consistently appear in the top 10 across all partitions, validating the robustness of 
    GBT-based feature selection for LSTM sensor reduction.}
    \label{fig:gbt_feature_stability}
\end{figure*}

Cross-validation analysis confirmed that these rankings remained stable across different 
data partitions, 
with 8 of the top 10 sensors appearing consistently across all five folds. 
The top five sensors, identified by their average importance, were used for the LSTM analysis. 
This cross-validated selection procedure ensures that the chosen sensors represent genuinely informative 
features rather than artifacts of a particular train-test split.

\subsubsection{Long Short-Term Memory}
\label{sec:lstm}

While GBT effectively captures batch-level patterns through statistical aggregation, it 
discards all temporal information. Faults in bioprocesses often manifest as 
dynamic signatures like unusual spikes, gradual drifts, or abnormal oscillations that 
occur at specific times during the batch. LSTM networks were selected to capture these 
sequential patterns over time.

A valid concern with semi-supervised anomaly detection is whether training exclusively on 
normal batches causes the model to flag normal process variability as faults.
The dataset includes a "Fault flag" column that indicate when a fault was injected during a run.
Each sequence from faulty batches are labeled as either pre-fault or post-fault. 
This approach makes it possible to verify that the model reacts to faults.

\begin{itemize}
    \item \textbf{Pre-fault sequences}: Both input window and forecast occur before fault 
    injection. These represent normal operation and should not trigger alerts.
    
    \item \textbf{Post-fault sequences}: The forecast horizon includes timesteps where the 
    fault is actively affecting process behavior. These should be detected as anomalies.
\end{itemize}

This separation enables rigorous measurement of both false positive rate (on held-out 
normal test batches) and true positive rate (on post-fault sequences).

\begin{figure}[!htbp]
    \centering
    \includegraphics[width=\columnwidth]{DO2_timestamp.png}
    \caption{LSTM 50-step ahead prediction for Dissolved Oxygen Concentration sensor on a 
    normal test batch. The model accurately captures the gradual increase in concentration 
    typical of the fermentation growth phase.}
    \label{fig:DO2_timestamp}
\end{figure}

The LSTM model trained on sequences from the normal batches only. A sliding window approach was used
to create input-output pairs for training. Each input consists of 100 consecutive timesteps, and each
output consists of 50 subsequent timesteps for all the five selected sensors.
This setup allows the model to learn to forecast future values based on historic data and it learns
the typical temporal patterns of processes under normal conditions.

During evaluation, the fault flag was used again so that sequences influenced by errors contributed to anomaly detection.

The LSTM architecture consists of two stacked layers (32 units each) followed by a dense 
output layer, with 20\% dropout regularization to prevent overfitting. The network was 
trained using the Huber loss function with the Adam optimizer.

For anomaly detection, the trained model predicts expected sensor values under normal 
operation. The absolute difference between predictions and actual measurements serves as 
the anomaly score. Detection thresholds are defined as 3 standard deviations above the 
baseline prediction error observed on normal test batches.

\begin{figure}[!htbp]
    \centering
    \includegraphics[width=\columnwidth]{LSTM_performance.png}
    \caption{Training and validation loss curves demonstrate model convergence without 
    overfitting.}
    \label{fig:LSTM_performance}
\end{figure}

Standardization was performed by fitting the scaler on training data only, and applied 
to validation and test sets to prevent data leakage.

Two mechanisms controlled the training process. Early stopping monitored validation 
performance with patience of 10 epochs, restoring the best model weights when improvement 
ceased. Learning rate reduction lowered the rate by 20\% when validation 
loss plateaued, with patience of 5 epochs.

\section{Results}
\subsection{Gradient Boosted Trees Results}

\subsubsection{Model Performance and Cross-Validation}

The XGBoost classifier, trained on 19 online sensors,
demonstrated strong performance. Cross-validation confirmed model robustness despite 
dataset limitations.

\textbf{Cross-Validation Performance (5-Fold Stratified):}
\begin{itemize}
    \item Validation Accuracy: $0.906 \pm 0.029$
    \item Validation F1-Score: $0.393 \pm 0.335$
    \item Validation AUC-ROC: $0.901 \pm 0.104$
\end{itemize}

\textbf{Test Set Performance:}
\begin{itemize}
    \item Test Accuracy: $0.867$
    \item Test F1-Score: Variable across folds
\end{itemize}

Cross-validation reveals the model successfully detects faults (90.6\% accuracy, 90\% 
AUC), though performance varies by fold because of limited faulty batch representation. When 
validation folds contain sufficient faulty samples (Folds 4-5), F1-scores reach 0.67 to 
0.80, indicating strong minority class detection capability.

Table \ref{tab:gbt_cv_folds} presents the detailed cross-validation results. The high 
variance in F1-scores ($\sigma = 0.33$) reflects stratified sampling creating validation 
folds with varying numbers of faulty batches rather than model instability. When 
sufficient faults are present (Folds 4-5), the model achieves F1 $\geq 0.67$, 
demonstrating effective fault detection.

\begin{table}[h]
\centering
\caption{Cross-validation results per fold}
\label{tab:gbt_cv_folds}
\begin{tabular}{cccc}
\hline
\textbf{Fold} & \textbf{Accuracy} & \textbf{F1-Score} & \textbf{AUC} \\
\hline
1 & 0.882 & 0.000 & 0.938 \\
2 & 0.882 & 0.500 & 0.933 \\
3 & 0.882 & 0.000 & 0.933 \\
4 & 0.941 & 0.667 & 0.700 \\
5 & 0.941 & 0.800 & 1.000 \\
\hline
\end{tabular}
\end{table}

Despite F1 variance, AUC remains consistently high (mean 0.90), confirming the model's 
reliable ranking ability. The model correctly scores faulty batches higher than normal 
ones, even when absolute fault counts are low in specific folds.

\subsubsection{Feature Importance and Sensor Selection}

Cross-validation across five folds identified the most critical sensors for fault 
detection. Table \ref{tab:gbt_top_sensors} shows the top five sensors ranked by mean 
importance.

\begin{table}[h]
\centering
\caption{Top 5 most important sensors (cross-validation average)}
\label{tab:gbt_top_sensors}
\begin{tabular}{lcc}
\hline
\textbf{Sensor} & \textbf{Mean} & \textbf{Std} \\
\hline
Temperature (T) & 0.166 & 0.101 \\
Acid flow rate (Fa) & 0.166 & 0.082 \\
Dissolved Oxygen (DO$_2$) & 0.144 & 0.098 \\
pH & 0.119 & 0.053 \\
Vessel Volume (V) & 0.065 & 0.049 \\
\hline
\end{tabular}
\end{table}

Temperature and Acid flow rate consistently rank in the top three across all folds, 
demonstrating robust feature selection. The moderate standard deviations reflect natural 
variation in feature utility across different data splits rather than model instability. 
Figure \ref{fig:gbt_feature_stability} illustrates sensor importance consistency across 
individual folds and their mean.

\subsubsection{ROC Analysis}

The cross-validated ROC curve (Figure \ref{fig:gbt_cv_roc}) demonstrates strong 
discrimination capability with mean AUC = 0.901 $\pm$ 0.104. Individual fold curves show 
expected variation given approximately 10 to 15 faulty samples per fold. Even the worst 
performing fold (Fold 4, AUC = 0.70) outperforms random guessing, while the best fold 
(Fold 5, AUC = 1.00) demonstrates potential when sufficient fault samples are available.

\begin{figure}[!htbp]
    \centering
    \includegraphics[width=\columnwidth]{GBT_CV_5.png}
    \caption{Cross-validated ROC curves across 5-fold stratified partitioning.}
    \label{fig:gbt_cv_roc}
\end{figure}

\subsubsection{Model Validation}

Test accuracy (0.867) falls within the cross-validation confidence interval (0.906 $\pm$ 
0.029), confirming that test set results are reliable and not due to fortunate data 
splitting. The model generalizes well to unseen data, and cross-validation accurately 
estimates real-world performance.

\subsection{LSTM Anomaly Detection Results}

\subsubsection{Model Architecture and Training}

The LSTM network was trained exclusively on 90 normal batches to learn expected process 
behavior patterns. The architectural details are discussed in Section \ref{sec:lstm}.
the results showed that the modle training converged after 18 epochs with early stopping, achieving a final validation loss 
of 0.089 (Huber). Figure \ref{fig:LSTM_performance} demonstrates stable convergence 
without overfitting, with training and validation losses remaining closely aligned 
throughout training.

\subsubsection{Anomaly Detection Performance and Threshold Selection}

Since the model was trained only normal data, the anomaly reliability depends 
on process deviation from learned behavior.
Using the sequence labeling mentioned in Section \ref{sec:lstm}, the model showed a clear separation
between normal and faulty sequences. Thresholds were selected from the validation error distribution.
Table \ref{tab:lstm_threshold_selection} summarizes well the trade-off. The 90th percentile provided the best balance,
detecting 43.1\% of post-fault sequences while maintaining a low false positive rate of 3.7\% on normal test sequences.

\begin{table}[h]
\centering
\caption{LSTM detection performance at different threshold percentiles. FPR measured on 
normal test sequences; TPR measured on post-fault sequences.}
\label{tab:lstm_threshold_selection}
\resizebox{\columnwidth}{!}{
\begin{tabular}{cccc}
\hline
\textbf{Threshold} & \textbf{FPR} & \textbf{TPR} & \textbf{FNR} \\
\textbf{Percentile} & \textbf{(\%)} & \textbf{(\%)} & \textbf{(\%)} \\
\hline
99th & 0.0 & 8.8 & 91.2 \\
97.5th & 0.1 & 29.3 & 70.7 \\
95th & 2.3 & 35.9 & 64.0 \\
\textbf{90th} & \textbf{3.7} & \textbf{43.1} & \textbf{56.9} \\
85th & 5.5 & 48.7 & 51.3 \\
80th & 8.3 & 54.4 & 45.6 \\
\hline
\end{tabular}
}
\end{table}

\subsubsection{Prediction Accuracy on Normal Batches}

The model achieved acceptable predictive performance on held-out normal test batches, with 
Mean Absolute Error (MAE) = 2.47 units and Root Mean Squared Error (RMSE) = 4.13 units 
across all five sensors.

Figure \ref{fig:lstm_predictions_temp} and \ref{fig:lstm_predictions_ph} illustrate 
representative 50-step predictions for two critical sensors: Temperature and pH. The LSTM
captured the general trends and dynamics of normal conditions.

\begin{figure}[htbp]
    \centering
    \includegraphics[width=\columnwidth]{Temp_timestamp.png}
    \caption{LSTM 50-step ahead predictions for Temperature. Green markers: actual 
    measurements; red crosses: LSTM predictions.}
    \label{fig:lstm_predictions_temp}
\end{figure}

\begin{figure}[htbp]
    \centering
    \includegraphics[width=\columnwidth]{pH_timestamp.png}
    \caption{LSTM 50-step ahead predictions for pH. Green markers: actual measurements; 
    red crosses: LSTM predictions.}
    \label{fig:lstm_predictions_ph}
\end{figure}

\subsubsection{Anomaly Detection Performance}

The LSTM model predicted expected "normal" trajectories for the 10 faulty batches never 
seen during training. Anomaly scores were calculated as absolute deviations between actual 
faulty measurements and predicted normal behavior.

Figure \ref{fig:lstm_anomaly_detection} demonstrates detection capability across all five 
sensors. 

\begin{figure}[htbp]
    \centering
    \includegraphics[width=\columnwidth]{normalvs_fault.png}
    \caption{Anomaly detection performance across sensors. Green: normal batch errors; 
    red: faulty batch deviations; orange dashed: threshold.}
    \label{fig:lstm_anomaly_detection}
\end{figure}

Temperature emerges as the most reliable early warning sensor, with anomalies detectable 
at 15.7 timesteps ahead (3.14 hours) and 57.0\% detection reliability across faulty 
batches. Dissolved Oxygen provides similar early detection (16.0 timesteps ahead) but with 
lower reliability (27.5\%). pH shows intermediate performance with 18.7 timesteps ahead 
detection and 48.1\% reliability.

\subsubsection{Early Warning Performance}

Table \ref{tab:lstm_detection_timing} summarizes detection timing for each sensor. The 
model provides alerts 3.1 to 4.1 hours ahead when anomalies exceed the threshold, 
enabling intervention substantially earlier than end-of-batch quality testing.

\begin{table}[h]
\centering
\caption{Early warning performance by sensor}
\label{tab:lstm_detection_timing}
\resizebox{\columnwidth}{!}{
\begin{tabular}{lccc}
\hline
\textbf{Sensor} & \textbf{Detection} & \textbf{Reliability} & \textbf{Priority} \\
 & \textbf{(hrs ahead)} & \textbf{(\%)} & \\
\hline
Temperature (T) & 3.14 & 57.0 & Critical \\
DO$_2$ & 3.20 & 27.5 & Secondary \\
pH & 3.74 & 48.1 & High \\
Acid flow (Fa) & 4.08 & 39.7 & Secondary \\
Vessel Volume (V) & 3.56 & 28.7 & Secondary \\
\hline
\end{tabular}
}
\end{table}

Using only the 5 GBT-selected sensors, the LSTM achieves robust anomaly detection with an average detection reliability of 40.2\%. 
This represents a 74\% reduction in monitoring complexity compared to the full 19 online sensor. This validates the sensor importance rankings and 
demonstrates that targeted monitoring suffices for robust fault detection.

\subsection{Comparative Analysis}

The GBT and LSTM models provide complementary capabilities for fault detection in 
penicillin fermentation. GBT excels at discriminating faulty from normal batches through 
statistical feature aggregation, achieving 90.6\% cross-validated accuracy and reliably 
identifying critical sensors (Temperature, Acid flow, DO$_2$). However, its reliance on 
summary statistics limits temporal resolution and early warning capability.

In contrast, the LSTM model leverages sequential patterns to predict future sensor 
trajectories, enabling fault detection before batch completion. While 
individual sensor detection rates are moderate, the temporal 
forecasting approach identifies deviations from expected normal behavior earlier in the 
process timeline.

The optimal monitoring strategy integrates both approaches: GBT for robust batch-level classification and sensor prioritization, and LSTM for real-time early warning across the five critical sensors identified by GBT. 
This integrated framework reduces monitoring complexity while delivering high-confidence fault classification and actionable early-intervention windows. 

\section{Integration Analysis}
\label{sec:integration}

Deploying machine learning for fault detection in pharmaceutical manufacturing is not only 
a question of model performance. Even highly accurate algorithms provide limited value if 
they cannot be integrated into existing production systems, regulatory frameworks, and 
daily operational workflows. This section examines how the proposed GBT and LSTM models 
integrate into real manufacturing environments, focusing on system architecture, data 
flow, and practical deployment challenges.

\subsection{Manufacturing Information System Hierarchy}

Pharmaceutical manufacturing systems are commonly structured according to the ISA-95 
standard, which defines a hierarchy from physical process equipment to enterprise-level 
planning systems [10]. Understanding this hierarchy is essential for placing machine 
learning models where they can deliver value without disrupting validated operations.

\begin{figure}[hbtp]
    \centering
    \includegraphics[width=\columnwidth, height = 140.0pt]{ISA-95.jpg}
    \caption{ISA-95 automation hierarchy}
    \label{fig:isa95_pyramid}
\end{figure}

At Levels 0--1, sensors directly interact with the fermentation process. 
These devices measure key process variables like temperature, pH, dissolved oxygen, and 
flow rates, continuously generating time-series data that reflect real process behavior 
[9, 11].

Level 2 consists of PLCs and SCADA systems responsible for real-time supervisory control. 
This layer maintains process setpoints and reacts to immediate deviations. The LSTM model naturally integrates at this level, as it processes 
streaming sensor data and generates anomaly predictions every 12 minutes (one timestep), 
enabling early detection while the batch is still running [8, 10].

Manufacturing Execution Systems (MES) at Level 3 coordinates batch execution, material tracking, and quality management [14]. '
This makes them better suited for batch-level intelligence rather than real-time control. The GBT 
classifier integrates at this level, providing fault probability estimates at predefined 
batch milestones to support quality investigations and production decisions.

Enterprise Resource Planning (ERP) systems at Level 4 manage inventory, procurement, and 
supply chain planning. While direct ML integration at this level is limited, aggregated 
fault statistics and quality trends can inform higher-level planning and capacity decisions 
[10].

Together, these layers highlight the complementary roles of the two modeling approaches. 
The LSTM provides tactical, real-time early warnings for operators, while the GBT delivers 
strategic batch-level insights for quality engineers and production managers. Both are 
required for effective industrial deployment.

\subsection{Integration Architecture and Data Flow}

Effective deployment requires a robust data pipeline that connects shop-floor 
instrumentation to computational infrastructure and decision-support systems. 
Figure~\ref{fig:mlops_architecture} illustrates a reference MLOps architecture adapted for 
pharmaceutical manufacturing [12].

\begin{figure*}[htbp]
    \centering
    \includegraphics[width=0.9\textwidth]{ML_Ops.jpeg}
    \caption{Example of MLOps architecture for pharmaceutical manufacturing from [12].}
    \label{fig:mlops_architecture}
\end{figure*}

On the shop floor, the five critical sensors identified through GBT feature selection 
(Temperature, Acid flow rate, DO$_2$, pH, and Vessel Volume) stream data to SCADA systems 
for process control. The same data is forwarded to centralized storage, where it supports 
both real-time LSTM inference and periodic retraining.

\textbf{Real-Time LSTM Pipeline:}  
The LSTM operates as a predictive soft sensor, every 12 minutes, new predictions are generated. 
Anomalies are detected by comparing predicted and observed values, with alerts triggered when deviations exceed a
threshold. These alerts are transmitted to the MES via standard industrial interfaces such 
as APIs [8, 12].

The MES enriches alerts with batch context: product recipe, batch identifier, and 
production stage. Operators receive concise, actionable information through application interfaces, 
showing which sensor deviated, by how much, and potential corrective actions based on 
process knowledge.

\textbf{Batch-Level GBT Integration:}  
GBT predictions are generated at predefined batch checkpoints (25\%, 50\%, and 75\% 
completion). At each checkpoint, the MES aggregates sensor statistics and computes a fault 
probability. Early identification of high-risk batches enables targeted interventions 
like increased sampling, additional offline analysis, or early batch termination, reducing 
waste and operational costs [5, 14].

\subsection{Machine Learning Operations in Regulated Environments}

Pharmaceutical manufacturing is subject to strict regulatory conditions, requiring ML 
systems to be fully traceable, validated, and auditable. MLOps practices provide the 
necessary framework to meet these requirements [12].

\textbf{Model Versioning and Traceability:}  
Each prediction must be traceable to a specific model version, training dataset, and 
validation outcome. This ensures compliance with FDA 21 CFR Part 11 requirements for 
electronic records and supports post hoc investigations when model-driven decisions affect 
product quality [9, 12].

\textbf{Performance Monitoring and Retraining:}  
Model performance is continuously monitored to detect degradation caused by process drift 
or equipment changes. While retraining pipelines can be automated, updated models must 
undergo formal validation and change control before deployment in production environments 
[12].

\textbf{Staged Deployment and Shadow Mode:}  
New model versions are typically deployed in shadow mode, where predictions are logged but 
not acted upon. This allows direct comparison with the production model without 
operational 
risk. Only after demonstrating equivalent or improved performance is the new model 
promoted 
to active use.

\subsection{Practical Deployment Challenges}

Several non-technical factors limit widespread adoption of ML-based fault detection in 
pharmaceutical manufacturing.

First, regulatory validation introduces significant time and cost overhead. Installation, 
operational, and performance qualification protocols often extend deployment timelines to 
6-12 months, with costs substantially higher than in non-regulated industries [9].

Second, many facilities rely on legacy infrastructure with limited data accessibility and 
interoperability. Retrofitting sensors, upgrading historians, and standardizing data 
pipelines can require substantial capital investment [8, 9].

Third, operator trust remains a critical factor. While GBT feature importance improves 
interpretability, LSTM-based anomaly detection is less transparent. Without clear 
explanations, operators may ignore alerts or treat them as nuisance alarms. Building trust 
requires clear interfaces, training, and demonstrated reliability over extended operation 
[14, 15].

Finally, integration with commercial MES platforms is complex because of validated 
configurations and restricted workflows. While service-oriented and multi-agent MES 
architectures show promise, they remain largely experimental in industrial settings [15].

\subsection{Bridging Research and Industrial Practice}

The gap between academic research and industrial deployment reflects differing priorities. 
Research emphasizes predictive accuracy and early detection, while industry requires 
compliance, reliability, maintainability, and seamless integration with existing systems.

This work addresses both perspectives. The proposed LSTM achieves robust anomaly detection 
while the GBT classifier reaches accurate batch-level accuracy. 
Beyond performance, the integration analysis demonstrates how these models 
align with ISA-95 hierarchies, MLOps practices, and real operational constraints.

Future work should validate these findings through pilot deployments in operational 
facilities, evaluating not only prediction accuracy but also intervention effectiveness, 
operator acceptance, and measurable improvements in product quality and cost efficiency 
[9, 12]. Ultimately, successful Industry 4.0 adoption in pharmaceutical manufacturing depends 
as much on sociotechnical integration as on algorithmic performance.

\section{Discussion}

The main goal of this research was to see if machine learning models can detect faults in
a pharmaceutical environment with a reduced set of sensors while still providing reliable insights.
Reducing the amount of sensors used was crucial, since sensors can be expensive in terms of 
installation, maintenance, calibration and operational complexity.

The structure of the IndPenSim V3 dataset introduced a serious challenge due to the
severe class imbalance of only 10 faulty batches out of 100 total.
This imbalance made it difficult to rely on standard classification approaches, because missing one or two faulty batches
could cause a significant drop in metrics like F1-score.
This was addressed by using AUC-ROC as the main metric for the GBT classifier, since it is less sensitive to
threshold selection and it gives a more stable indication of how well the model can rank sensors by their importance and 
by how much they contribute to fault detection.

The GBT results were solid, the model achieved a cross-validated accuracy of 90.6\% and an AUC of 0.901.
The feature importance were consistent throught the folds with temperature, acid flow rate and dissolved oxygen levels being
the top three most important sensors.
The consistency between different folds validated the these sensors are the most informative for fault detection and that they
can be used for the LSTM model.

The LSTM model was trained only on normal batches with the top 5 sensors extracted from the GBT feature importance and
it was used to identify faulty patterns in unseen faulty batches.
The model achieved a detection rate of 43.1\% on post-fault sequences, with low false rate of 3.7\% on normal data.
While this detection doesn't seem very high, it is important to remember that each batch generates
multiple overlapping sequences, so the chnaces of detecting a fault at the batch level are much higher.
Moreover, the early warning times ranged between 3-4 hours before batch completion,
which is a a significant lead time for operators and engineers to take corrective actions.

A serious concern about the semi-superived approach of the LSTM model was raised about whether the model is able to 
distinguish between normal and faulty batch correctly. 
This concern was addressed by separating the sequences into pre-fault and post-fault categories using the "Fault flag" column in the dataset
during the evaluation of the model.
The low false positive rate demonstrates that the model is correctly identifying patterns that deviate from normal behavior rather
than just flagging normal variability.

There are still limitations to this work. The main one being the dataset. 
With only 10 simulated faults, it is difficult to claim the model can generalize well in a real manufacturing environment.
By focusing only on 5 sensors, the model may have missed some important signals whihch could have improved the forecasting accuracy and the detection rate.
These choices were made deliberately to reflect practical constraints in real pharmaceutical manufacturing,
where installing and maintaining a large number of sensors can be costly and operationally complex.

Another major limitation is that it wasn't possible to validate the framework in a real MES environment due to the proprietary nature of these systems.
The integration analysis is based only on literature and theoritical frameworks, without any empirical validation. 
As a result, the practical challenges about real-time performance, operator interaction, and regualatory compliance 
remain unaddressed and need to be explored in the future.

Despite these challenges, this work makes a meaningful contribution. It demonstrates that it's possible to get meaningful 
early warnings about faults using a smaller set of sensors, which can help reduce costs and complexity.
The results shows that a targeted, cost-effective monitoring strategy doesn't completely sacrify all detection capability.
It also shows, that with careful design and validation, it is possible to find some level of balance between performance and practicality,
which is essential in real-world regulated industries.

Overall, this research provides a solid foundation for future work in pharmaceutical manufacturing and machine learning. The models
worked relatively well under experimental conditions, however, more work is needed to validate these findings by using more realistic datasets,
deploying in real manufacturing environments.
Still, the findings provided a realistic and practical step toward smarter, and more cost-effective process monitoring.

\section{Conclusion}

Pharmaceutical manufacturing faces the dual challenge of maintaining stringent quality 
standards while adopting Industry 4.0 technologies to improve efficiency and reduce waste. 
This research explored whether the application of machine learning models can support 
earlier detection in faulty batches in a pharmaceutical process, using a 
reduced set of critical sensors for meaningful monitoring.

Using a simulated penicillin fermentation dataset, two complementary approaches were used:
Gradient Boosted Trees for batch-level fault classification and sensor ranking, and Long Short-Term 
Memory forecasting for real-time anomaly detection. These two methods adressed different aspects of this project
and they provided a balanced view of what is possible under practical constraints.
GBT reached a cross-validated accuracy of 90.6\% and AUC of 0.901, identifying Temperature, Acid flow rate, and Dissolved Oxygen as the most informative sensors.
LSTM networks, trained only with normal batches and the 5 sensors selected by gbt, achieved a modest detection rate. This lowered
monitoring complexity while still providing early warning 3-4 hours before batch completion.

Beyond model performance, this work highlighted the methodological consideration for highly imbalanced industrial datasets.
Cross-validated feature importance helped reduce sennsitivity to data partitioning, while the semi-supervised
approach avoided the instability of supervised classification with few faulty samples.
The sliding-window sequence labeling enabled a solid evaluation of false positive and true positive rates by
explicitly separating pre-fault and post=fault sequences, resulting in 3.7\% false positive rate and 43.1\% detection rate.

At the same time, substantial limitations remain. The dataset includes only 10 simulated 
faults, preventing confident claims about generalization or fault coverage. The moderate 
sequence-level detection rate (43.1\%) reflects the challenge of detecting diverse fault 
types with limited examples, though batch-level detection is higher due to multiple 
overlapping sequences per batch. More importantly, integration with real MES systems, 
regulatory validation, and operator interaction were not implemented due to the 
proprietary nature of pharmaceutical manufacturing execution systems and the lack of 
access to validated production environments.
The integration analysis discussed in \ref{sec:integration} is grounded in ISA-95 industrial practice but it remains 
a theoretical framework without tested validation. These gaps highlight the space that divides
academic research from industrial deployment, where practical implementation requires industry partnerships
and access to production systems.

Future work should address several directions that would facilitate the move of pharmaceutical fault detection with
smaller sensor sets closer to practical deployment.
First, the validation of richer datasets with more fault types from real manufacturing environmnets would
provide stronger evidence of generalization and robustness. Real data would also reveal false alarm rates remain acceptable
under true process variability, equipment drift, and batch-to-batch differences that are difficult to simulate.

Second, the deployment of this framework into an operational facility is essential to validate the integration analysis discussed 
in Section \ref{sec:integration}. This includes real-time data pipelines, communication protocols 
between ML models and SCADA/MES systems. Such deploymnets would also allow measurement
of impact on the company, such as reductions in batch failures, earlier interventions, and cost savings.

Third, hybrid modeling approaches that combine physics-based process understanding with data-driven learning 
could improve detection accuracy and interpretability. 
LSTM models learn only from known features' historical data, generalizing patterns seen during training,
however, it doesn't have an understanding of the underlying process dynamics. 
Theory-Trained Neural networks (or Physics-informed Neural Networks) could incorporate the knowledge 
of the fermentation process into the learning process, potentially improving the accuracy and roebustness of fault detection.

Finally, improving model interpretability remains essential for regulatory acceptance. 
GBT feature importance is straightforward, however, LSTM detections are less transparent.
Techniques such as attention mechanism, SHAP values could help engineers understand why a sequence was flagged,
building trust and facilitating regulatory validation.

The central conclusion is measured but optimistic. Machine learning can support
pharmaceutical fault detection by enabling earlier intervention and more focused
monitoring with reduced infrastructure. However, successful deployment depends a variaty of factors. 
This research provides a realistic method, integration framework, and an honest
assessment. It does not solve pharmaceutical fault
detection but clarifies the path forward and the work still required to bring
Industry 4.0 concepts into validated production environments.

\newpage
\onecolumn
\section*{References}

\vspace{0.3cm}
\noindent
[1] Katta, S. R. (2023). Predictive machine learning models for calibration 
failure detection in pharmaceutical manufacturing.
\textit{Journal of Artificial Intelligence, Machine Learning and Data 
Science}, 1(1), 2152--2160. \url{https://doi.org/10.51219/JAIMLD/Srikanth-reddy-katta/472}

\vspace{0.3cm}
\noindent
[2] Lawrence, N. P., Damarla, S. K., Kim, J. W., Tulsyan, A., Amjad, 
F., Wang, K., et al. (2024). Artificial intelligence applications 
for fault detection. \textit{Control Engineering Practice, 145}, 
105841. \url{https://doi.org/10.1016/j.conengprac.2024.105841}

\vspace{0.3cm}
\noindent
[3] Qiu, K., Wang, J., Zhou, X., Guo, Y., \& Wang, R. (2020). 
Prediction of machine failure in Industry 4.0: A hybrid CNN-LSTM 
framework. \textit{Industrial \& Engineering Chemistry Research, 59}
(44), 19633--19642. \url{https://doi.org/10.1021/acs.iecr.0c03806}

\vspace{0.3cm}
\noindent
[4] Yan, W., Tang, D., \& Lin, Y. (2017). Prediction of manufacturing 
processes errors using deep learning-based soft sensors. 
\textit{IEEE Transactions on Industrial Electronics, 64}(5), 4237-
-4245. \url{https://doi.org/10.1109/TIE.2016.2622668}

\vspace{0.3cm}
\noindent
[5] Acosta-Pavas, J. C., Robles-Rodriguez, C. E., Daboussi, F., 
Aceves-Lara, C. A., \& Corrales, D. C. (2024). Leveraging random 
forests and gradient boosting for enhanced predictive analytics in 
operational efficiency. \textit{Computers and Chemical Engineering, 
187}, 108736. \url{https://doi.org/10.1016/j.compchemeng.2024.108736}

\vspace{0.3cm}
\noindent
[6] Bayram, F., Ahmed, B. S., \& Kassler, A. (2022). Tackling faults 
in the Industry 4.0 era: A survey of machine-learning solutions and 
key aspects. \textit{Knowledge-Based Systems, 245}, 108632. 
\url{https://doi.org/10.1016/j.knosys.2022.108632}

\vspace{0.3cm}
\noindent
[7] Vijayaraj, N., Rajalakshmi, D., Immaculate, P. S., 
Sathianarayani, B., Rajeswari, S., \& Gomathi, S. (2024). An 
innovative approach to improve the quality of pharmaceutical 
manufacturing using cloud computing. \textit{EAI Endorsed 
Transactions on Pervasive Health and Technology, 10}(1), 1--6. 
\url{https://doi.org/10.4108/eetpht.10.5270}

\vspace{0.3cm}
\noindent
[8] Parapalli, S. L., \& Shetty, J. (2025). Leveraging hybrid edge-
cloud predictive maintenance in pharmaceutical MES: An Industry 4.0 
approach using big data. \textit{Journal of Computer Science and 
Technology Studies, 7}(2), 86--94. 
\url{https://doi.org/10.32996/jcsts.2025.7.2.7}

\vspace{0.3cm}
\noindent
[9] Arden, N. S., Fisher, A. C., Tyner, K., Yu, L. X., Lee, S. L.,\& 
Kopcha, M. (2021). Industry 4.0 for pharmaceutical manufacturing: 
Preparing for the smart factories of the future. 
\textit{International Journal of Pharmaceutics, 602}, 120554. \url{https://doi.org/10.1016/j.ijpharm.2021.120554}

\vspace{0.3cm}
\noindent
[10] Do, Y., \& Jeong, J. (2023). Designing a cloud-based MES-SaaS 
platform model in precision manufacturing. \textit{WSEAS 
Transactions on Computer Research, 11}, 294--302. \url{https://doi.org/10.37394/232018.2023.11.27}

\vspace{0.3cm}
\noindent
[11] Goldrick, S., Duran-Villalobos, C. A., Jankauskas, K., Lovett, 
D., Farid, S. S., \& Lennox, B. (2019). Modern day monitoring and 
control challenges outlined on an industrial-scale benchmark 
fermentation process. \textit{Computers and Chemical Engineering, 
130}, 106471. \url{https://doi.org/10.1016/j.compchemeng.2019.05.037}


\vspace{0.3cm}
\noindent
[12] Metcalfe, B., Acosta-Pavas, J. C., Robles-Rodriguez, C. E., 
Georgakilas, G. K., Dalamagas, T., Aceves-Lara, C. A., Daboussi, F., 
Koehorst, J. J., \& Corrales, D. C. (2025). Towards a machine 
learning operations (MLOps) soft sensor for real-time pharmaceutical 
manufacturing. \textit{Computers and Chemical Engineering, 194}, 
108991. \url{ https://doi.org/10.1016/j.compchemeng.2024.108991}

\vspace{0.3cm}
\noindent
[13] Goldrick, S. (2019, October). Data for: Modern day monitoring and control challenges 
outlined on an industrial-scale benchmark fermentation process (Version 2) [Dataset]. 
\textit{Mendeley Data}. \url{https://doi.org/10.17632/pdnjz7zz5x.2}

\vspace{0.3cm}
\noindent
[14] Shojaeinasab, A., Charter, T., Jalayer, M., Khadivi, M., Ogunfowora, O., Raiyani, N., 
Yaghoubi, M., \& Najjaran, H. (2022). Intelligent manufacturing execution systems: A 
systematic review. \textit{Journal of Manufacturing Systems}, \textit{62}, 503--522. \url{https://doi.org/10.1016/j.jmsy.2022.01.004}

\vspace{0.3cm}
\noindent
[15] Mantravadi, S., Li, C., \& Møller, C. (2019). Multi-agent manufacturing execution 
system (MES): Concept, architecture \& ML algorithm for a smart factory case. In 
\textit{Proceedings of the 21st International Conference on Enterprise Information Systems 
(ICEIS 2019)} (pp. 477--482). SCITEPRESS. \url{https://doi.org/10.5220/0007768904770482}


\end{document}
